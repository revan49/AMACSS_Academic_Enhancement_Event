\documentclass[12pt]{article}
\begin{document}
This is where you put your answers for you questions. 

Answers for CSCA48:

Question 1:
You have a list of integers, and randomFunc returns a list where each index  
is the product of every integer at every index except for the current index.
Complexity is O(n^2)

Question 2: 

This is a post order traversal implemented iteratively.
  
  def inorder(root):
    if root != None:
        inorder(root.left)
        print(root.data)
        inorder(root.right)

Question 3:

def heapify(A, i, size):
	max = i
	if ((2i+1) <= size and A[2i+1] > A[i]):
		max = 2i+1
	if ((2i+2) <= size and A[2i+2] > A[max]):
		max = 2i+2
	if ( max != i ):
		# swap A[i] and A[max]
		temp = A[max]
		A[max] = A[i]
		A[i] = temp
		# recursively call heapify
		heapify(A, max, size)

Complexity:
We know that a heap is represented by a compact tree that has the height of floor(lgn)+1.
Therefore, the Big-Oh is O(lg(n)).


\end{document}
